\documentclass[12pt,a4paper]{article}
\usepackage[utf8]{inputenc}
\usepackage{amsmath, amssymb, amsthm}
\usepackage{mathrsfs}

\begin{document}

\begin{center}
\textbf{DM2 - A rendre en TD mardi 30/9 ou vendredi 3/10}
\end{center}

\bigskip

\noindent
\textbf{Problème : Les statistiques d'ordre.} Soit $n \geq 1$ un entier fixé. Étant donné un $n$-uplet $(x_1, \dots, x_n)$ de nombres réels, nous le réordonnons pour obtenir le $n$-uplet $(x_{(1)}, \dots, x_{(n)})$ vérifiant
\[
x_{(1)} \leq x_{(2)} \leq \cdots \leq x_{(n)}.
\]

\noindent
0) Que se passe-t-il si $x_1, \dots, x_n$ ne sont pas distincts ? Que valent $x_{(1)}$ et $x_{(n)}$ ?

\bigskip

\noindent
\textbf{Partie I : Résultats généraux}

\medskip

\noindent
Soit $\mu$ une loi de probabilité sur $\mathbb{R}$ qui est à densité, et soit $f$ une densité de $\mu$. Soit $X_1, \dots, X_n$ un échantillon de loi $\mu$, i.e., $X_1, \dots, X_n$ sont $n$ variables aléatoires définies sur un même espace de probabilité $(\Omega, \mathscr{F}, \mathbb{P})$, qui sont i.i.d. de loi $\mu$. Nous notons alors $(X_{(1)}, \dots, X_{(n)})$ le $n$-uplet obtenu en réordonnant $(X_1, \dots, X_n)$. La variable $X_{(k)}$ est appelée la $k$-ième statistique d’ordre d’un échantillon $X_1, \dots, X_n$.

\medskip

\noindent
1) Quelle est la probabilité pour que deux v.a. parmi $X_1, \dots, X_n$ soient égales ?

\medskip

\noindent
2) Notons $\mathfrak{S}_n$ le groupe symétrique des permutations de $\{1, \dots, n\}$. Montrez qu’avec probabilité $1$, il existe une unique permutation $\Sigma \in \mathfrak{S}_n$ telle que
\[
(X_{(1)}, \dots, X_{(n)}) = (X_{\Sigma(1)}, \dots, X_{\Sigma(n)}).
\]

\medskip

\noindent
3) La permutation $\Sigma$ définie en 2) est un élément aléatoire de $\mathfrak{S}_n$. Quelle est sa loi ?

\medskip

\noindent
4) Soit $\phi$ une fonction continue bornée de $\mathbb{R}^n$ dans $\mathbb{R}$.

\begin{itemize}
    \item[a)] Montrez que
    \[
    \phi(X_{(1)}, \dots, X_{(n)}) = \sum_{\sigma \in \mathfrak{S}_n} \phi(X_{\sigma(1)}, \dots, X_{\sigma(n)}) \, 1_{\{ \Sigma = \sigma \}}.
    \]

    \item[b)] Montrez que, pour tout $\sigma$ dans $\mathfrak{S}_n$, $X_{\sigma(1)}, \dots, X_{\sigma(n)}$ est un échantillon de loi $\mu$ et que
    \[
    \mathbb{E}\!\left(\phi(X_{(1)}, \dots, X_{(n)}) \, 1_{\{ \Sigma = \sigma \}}\right)
    = \mathbb{E}\!\left( \phi(X_1, \dots, X_n) \, 1_{\{ X_1 < \cdots < X_n \}} \right).
    \]

    \item[c)] Conclure finalement que
    \[
    \mathbb{E}(\phi(X_{(1)}, \dots, X_{(n)})) = n! \int_{\mathbb{R}^n} \phi(x_1, \dots, x_n) \, 1_{\{x_1 < \cdots < x_n\}} f(x_1) \cdots f(x_n) \, d\lambda(x_1) \cdots d\lambda(x_n).
    \]
\end{itemize}

\medskip

\noindent
5) Soit $g$ une fonction continue bornée de $\mathbb{R}$ dans $\mathbb{R}$. En appliquant la question 4) avec une fonction $\phi$ bien choisie, montrez que, pour tout $1 \leq k \leq n$,
\[
\mathbb{E}(g(X_{(k)})) = \frac{n!}{(k-1)! (n-k)!} \int_{\mathbb{R}} g(x) \, \bigg( \int_{-\infty}^x f(t) \, d\lambda(t) \bigg)^{k-1} \bigg( \int_{x}^{+\infty} f(t) \, d\lambda(t) \bigg)^{n-k} f(x) \, d\lambda(x).
\]

\noindent
En déduire que la loi de $X_{(k)}$ admet une densité que l’on explicitera.

\bigskip

\noindent
\emph{Anytime you have a 50-50 shot at getting something right,\\
there’s a 90\% probability you’ll get it wrong.}
\hfill \emph{Andy Rooney}

\end{document}

